\documentclass[onecolumn]{IEEEtran}
\title{Cache Based Side Channels: Attacks and Defenses}
\author{Sathya Chandran Sundaramurthy \\ \small{sathyachandr@mail.usf.edu}}
\begin{document}
\maketitle
\begin{abstract}
The efficiency of a processor relies heavily on the use of cache
memory.  However, the sharing of same cache space between different
processes opens up a wide possibility of attacks.  Research over the
years has exposed the use of side-channels as a new attack mechanism
to extract sensitive information from a targeted victim process. The
attack could be as severe as extraction of the private key of an
encryption algorithm.  My term paper will explore various types of
side-channel attacks possible on cache memory and the evolution of
defense techniques to mitigate the same.  This proposal provides a
brief overview of the problem, and a outline for the final submission.
Being a graduate student in Computer Science foucsing on security this
problem seemed to be an interesting one to write my term paper on.
\end{abstract}
\section{Introduction}

A cache is a block of fast RAM that sits inside a CPU die.  Whenever a
program refers to a memory address the processor first checks the
cache to see if the contents of that memory address have already been
loaded.  This case is called a {\it cache hit}.  If the contents are
not available in the cache yet, called a {\it cache miss}, the
contents are then loaded from memory onto the cache.  Subsequent
accesses to the same memory address will now be faster. There are
three categories of cache memory classified based on their proximity
to the CPU --{\it L1, L2, and L3} -- with L1 being closest.  Closer
the cache to the CPU, lesser the access time, and smaller the size.
In a modern processor. such as Intel Xeon, which has a
multi-core architecture, the L1 and L2 caches are private to each core
while the L3 cache is shared among all the cores.

Information can leak in a number of ways from a cache memory.  Let us take
a brief look at the various ways in which information disclosure can occur
in cache memory~\cite{kim2012stealthmem}.

\paragraph{Premptive scheduling} Consider the case of a CPU with a
single core in a virtualized environment.  Assume the single core is
allocated two VMs.  Now, since the CPU can execute only one VM at any
given time, it has to do periodic context switching between the two.
During the switch, the scheduled VM has access to the state of the
cache as left by the previous VM which leads to information leakage.

\paragraph{Hyper-Threading} In this technique two different threads
can execute simultaneously on a single CPU core.  This is achieved by
allowing the two threads to share all of the CPU resources such as
caceh, ALU etc.  There is much more opportunity for information
leakage in this case compared with the previous case.

\paragraph{Multicore} In a multicore architecture of CPU, each core
has a private L1 and L2 cache.  The L3 cache is shared among all the
cores.  This is a little restrictive environment for the attacker.
The adversary has only one possibility, which is to probe the L3 cache
to obtain information on the cache access pattern by the victim.

With this brief introduction to caches let us now look at the various
types of side-channel attacks that have been found to be possible on
cache memory.

\section{Attack Classification}

Cache based side-channel attacks fall under three broad categories:

\begin{itemize}
\item Access-driven Cache Attacks
\item Time-driven Cache Attacks
\item Trace-driven Cache Attacks
\end{itemize}

\subsection{Access-driven Cache Attacks}

A highly cited example for this type of attack is due to
Percival~\cite{percival2005cache}.  Let us take an example of an
adversary trying to recover the private key of AES encryption.  The
attacker and the encryption module are two different processes
executing on the same CPU core.  First, the attacker will try to fill
the entire cache with his own data by loading a large array.  He will
then let the encryption process to execute.  AES encryption makes use
of lookup tables which in turn depend on the plain text and the key.
Now, in order to do the encryption the appropriate table entries will
be loaded into the cache.  The loaded entries will replace some of the
entries previously loaded by the attacker.  After the encryption, the
attacker will again load the entire array.  He also measures the time
taken to load the array into the cache lines.  If the time taken to
fill a cache line takes longer then he knows that the entry has been
evicted by the encryption process and from that he can guess the index
of the lookup table.  Research has shown that it is possible to
extract the entire 128 bits of the AES key using this
attack~\cite{aciiccmez2007yet,osvik2006cache}.

\subsection{Time-driven Cache Attacks}

This type of attack relies on measuring the execution time of a
process for various inputs.  For example, in case of AES encryption
the total execution time of the process depends on the plain text and
key.  A common attack in this scenario is the ``cache collision
attack''.  This attack makes use of the experimental result that
higher number of cache collision leads to less cache misses which in
turn leads to shorter execution
times~\cite{osvik2006cache,bonneau2006cache,tiri2007analytical}.  By
measuring the execution time for encryption of large number of plain
texts using the same key the attacker identifies that one execution
which has the lowest mean execution time.  He then is able to deduce
the entire AES key based on the last round
attack~\cite{bonneau2006cache,tiri2007analytical}.

These attacks are classified further based on the position of the adversary:

\begin{itemize}
\item Passive time-driven cache attacks
\item Active time-driven cache attacks
\end{itemize}

\subsubsection{Passive time-driven cache attacks}

In a passive attack the attacker cannot directly probe or modify the
victim's cache.  The passive attacker also does not have access to
timing information on the victim's machine making the attack harder.
The inability to measure accurately the time and lack of direct access
to the cache creates noise in the attacker's measurements.  In spite
of this, the attack can be carried out using a large number of
samples, such as the Bernstein's attack~\cite{bernstein2005cache} to
recover the AES key using $2^{27.5}$ measurements.
 
\subsubsection{Active time-driven cache attacks}

In an active attack, the attacker has direct access to victim's cache
which enables him to cause collisions and even manipulate the state of
the cache.  The direct access also provides with the ability to
measure the time accurately unlike the passive case.  Osvik et
al.~\cite{osvik2006cache} demonstrate an attack on AES that recovers
the complete 128 bit key using 500,000 measurements which is much more
efficient than Bernstein's passive attack.

\subsection{Trace-driven Cache Attacks}

This attack, like active time-driven attack requires the attacker to
have direct access to victim's cache.  Specifically, the attacker
tries to observe the access pattern of cache lines by the victim and
tries to manipulate or probe the cache lines based on the observation
to his advantage.  A commonly cited example of this type of attack is
the Prime + Probe attack.  As a first step, the attacker fills the
cache lines with contents of certain memory address (Prime).  After
some time, the attacker tries to access the contents at the same
memory address (Probe).  If the time taken to access contents at the
probe stage is longer than the prime stage then the attacker infers
that the victim has accessed a pre-image (memory address) of the same
cache line as the attacker.  The leaked information here is the access
record of the victim process.

Attacks based on this idea have been successfully demonstrated by
researchers.  The attacks are made easier due to the Hyper-Threading
features available in processors where two threads share a number of
hardware resources executing simultaneously.
Percival~\cite{percival2005cache} exploited this idea and demonstrated
an attack on the RSA encryption where he was able to obtain the traces
of modular exponentiation and multiplication operations of the
algorithm.  In another work, Neve~\cite{neve2006advances} demonstrated
the feasibility of trace-driven attacks on a single-threaded processor
making it more severe and easier than the Hyper-Threading version.  In
yet another attack, Gullasch et al.~\cite{gullasch2011cache} were able
to extract the full key of AES encryption using Linux Scheduler.

The attacks describe in this section and before show the practical
feasibility and the ease of conducting cache based side channel
attacks.

\section{Defense Mechanisms}

Current mechanisms for defending against cache based side-channel
attacks are either {\it hardware based} or {\it software based}.
Hardware based solutions suggest a modification to existing cache
design.  This would involve a change in the chip manufacturing
process.  An example would be the work by Domnitser et
al.~\cite{domnitser2012non} who propose a new cache design called
Non-monopolizable caches (NoMo).  The basic idea is to reserve a
few cache lines for each process non-evictable.  The authors claim
that their approach requires only a slight modification to the
hardware's cache replacement algorithm with no other support.

Software based approaches, on the other hand, work by modification to
operating system kernel without requiring any changes to underlying
hardware.  Kim~\cite{kim2012stealthmem} et al. present an idea of
stealth memory where each virtual machine or a process is allocated
certain parts of cache lines that are never evicted by another
process.  This non-evictable cache memory is called stealth memory.
Since a process or VM cannot evict the locked cache lines side channel
attacks can be mitigated.  The solution is an extension of hypervisor
code in the case of VMs or operating system code in the case of single
machine(s).

The final term paper will include a detailed overview of all the software
and hardware based defense mechanisms proposed in the research literature.

\section{Proposed Outline for Final Paper}

In the full version of my term paper I plan to discuss the following:

\begin{itemize}
\item Evolution of cache based side-channel attacks in the research
literature
\item Comprehensive analaysis of various defense mechanisms -- pros
and cons
\item My own critique on the nature of the problem and proposed
solutions
\item Finally, expected trajectory of the problem that future
solutions should consider
\end{itemize}

\bibliographystyle{plain}
\bibliography{term-paper}
\end{document}